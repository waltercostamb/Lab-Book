 \documentclass[12pt,a4paper]{report}
 \usepackage[dvipdfmx]{graphicx}
 \usepackage{listings}
 \usepackage[pdftex]{hyperref}
 \usepackage{url}
 \usepackage{graphicx,color}
 \usepackage[font=scriptsize]{caption}
 \usepackage{subcaption}
 \usepackage{fullpage}
 \usepackage{footnote}
 \usepackage{pdfpages}
 \usepackage{amsmath}
 \usepackage{longtable}
 \usepackage{tcolorbox}

 \newcommand\tab[1][0.5cm]{\hspace*{#1}}
 \newcommand{\command}[1]{\textcolor{blue}{#1}}
 \newcommand{\titulo}{{\bf Lab Book}\\{\it project title}\\Author}
 \newcommand{\autor}{Advisor\\Institute}


\title{\titulo}
\author{\autor}
\date{2018}

 \begin{document}
 \maketitle
 %\listoffigures - List the figures
 %\listoftables - List the tables
 \tableofcontents
 \newpage

%%%%%%%%%%%%%%%%%%%%%%%%%%%%%%%%%%%%%%%%%%%%%%%%%%%%%
%%%%%%%%%%%%%%%%%%%%%%%%%%%%%%%%%%%%%%%%%%%%%%%%%%%%%

 \chapter{May 2018}
 
 \section{13}
 \subsection{Learning \LaTeX}
 \hspace{0.2cm}
 \begin{tcolorbox}[width=6.3in]
 \scriptsize 
 - Working folder: \textit{path}
 \end{tcolorbox}
 \hspace{0.2cm}
 \normalsize  
 
  \LaTeX{} is a high-quality typesetting system, available as free software, which allows to produce scientific or technical documents \cite{latex-main}. I am using \LaTeX{} to create a Bioinformatics Lab Book. To compile my Lab Book, I can use command lines (\command{pdflatex} and \command{bibtex}). Afterwards I can visualise the produced {\it .pdf} file with evince or another reader. Alternatevily, I can use a Latex editor, such as TexWorks (\url{https://www.tug.org/texworks/}), which allows me to write the code and control the {\it pdf} file in the same environment (Figure~\ref{texworks}).  \\
  
 % \newpage
  
  To compile the {\it .tex} file in the command line: \\
  
  \command{\$pdflatex lab-book}
  
  \command{\$bibtex lab-book}
  
  \command{\$pdflatex lab-book}
    
  \command{\$pdflatex lab-book} \\
  
   To visualise the {\it .pdf}: \\
  
  \command{\$evince lab-book.pdf \&}
  
    \begin{figure}
  \centering 
  \includegraphics[width=1.0\textwidth]{figures/texworks-linux.png} 
  \caption[TexWorks Editor.]{TexWorks editor (\url{https://www.tug.org/texworks/}) layout in a Linux machine.}
  \label{texworks} 
  \end{figure}
  
 \subsection{Math environment}
  This is the equation environment, which numbers equations: \\
  
  \begin{equation}
  F(x) = \int^a_b \frac{1}{3}x^3
 \end{equation}
 
  \newpage
 This is the align environment, without numbering equations (uses package {\it amsmath}): \\
 
  \begin{align*}
   f(x) &= x^2\\
   g(x) &= \frac{1}{x}\\
   F(x) &= \int^a_b \frac{1}{3}x^3
 \end{align*}
 
  \subsection{15 - Short-term project proposal}
 Some text here. Incluing and referencing a table (table~\ref{table1}).
 
 \begin{itemize}
\item First numbered list item
\item Second numbered list item
\end{itemize}

\begin{table}[!htb]
  \caption{table1}
  \centering
  \begin{tabular}{ccc}
  \hline 
       species&changes&score \\
  \hline
       Macaque&4&0.0 \\
       Human&2&14.9 \\
       Orangutan&0&0.0 \\
       Pan&0&0.0 \\
       Gorilla&0&0.0 \\
  \hline
  \end{tabular}
  \label{table1}
 \end{table}
 
 
 \bibliographystyle{apalike}
 \bibliography{ref}


 \end{document}
